\cbox{Section word count: ~1700}
\cbox{Needs work to make it flow better, also shortened}

The development of a `circular economy' has become a critical area of focus in the imperative pursuit of achieving sustainability objectives and curtailing our environmental footprint within planetary boundaries \citep{eu2019greendeal, eu2020circ,nl2023ceplan,nl2016ceplan,pardo2018ce,ellenmacarthur2015ce}. Fundamental to this development is a decrease in primary material consumption and a reduction of life cycle waste through the implementation of `re-X' strategies (e.g., refuse, rethink, design for---and implementation of---repair, remanufacturing and recycling) \citep{eu2022ecodesign, eu2022repair,eu2015reman}. In addition to circular economy goals, contemporary geo-political tensions in an ever more globalised economy have highlighted the vulnerability of many advanced economies to intentional supply disruptions, wrought as an act of competition or outright hostility \citep{jrc2023supplychain,hartley2024cepolitics,berry2023crm}.

While some material demands are apparent in the final product and the waste generated may be inferred from knowledge of the use-- and end-of-life-- (EOL) phases, a significant proportion of these are often `hidden' in the supply chain and thus not reported directly in the final results \citep{laurenti2016wastefootprint,salviulo2021supplychain}. It has been found that these material footprints can be `highly representative of damage to human health and biodiversity' \citep{steinmann2017resourcefootprints} and that waste footprints have a `strong association' with environmental damage \citep{laurenti2023wastefootprint}. Thus, to reduce the negative externalities of consumption and improve supply chain resilience, it is essential to uncover, disaggregate, and quantify the material and waste footprints of human activities in as much detail as possible.

Thus, we were motivated to create a flexible and policy-driven approach to understanding a product’s demand on the waste management system both now and in the future to ensure that models of circularity and improvements in technology and policy for waste are included in LCA modeling.

\subsection{LCA}

Life Cycle Assessment (LCA) is a useful method for the holistic estimation of the environmental impacts of products and processes. LCA can comprehensively evaluate these impacts across the entire life cycle---from `cradle to grave'---, often identifying critical hotspots and guiding prioritisation of actions. The standard approach is to apply Life Cycle Impact Assessment (LCIA) methods (such as ReCiPe \citep{huijbregts2016recipe} and CML \citep{guinee2002cml}), which convert the inventory data into a set of impact scores based on the sum of the elementary flows. These scores are then aggregated into a single score for each impact category, which can be compared across products and processes.

Several LCIA methods include, to some extent, waste generation \citep{foen2021ecofactors,hauschild2003edip,cen2019en15804} and material consumption \citep{arvidsson2020csi,foen2021ecofactors}. These methods, however, are generally limited in their scope (especially for waste), do not allow for flexible quantification of specific waste and material types, and often provide results in characterised units that are abstract or difficult to interpret (e.g., kg-Si equivalents or Ümweltbelastungspunkte (UBP)).


\subsection{Material Demand in LCA}
In the context of a mineral-hungry renewable energy transition and recent geo-political tensions, more attention is being paid to the security of supply of materials, especially those considered `critical raw materials' (CRMs) \citep{eu2023crmstudy,hool2023crm,mancini2013supplysecurity,jrc2023supplychain,hartley2024cepolitics,salviulo2021supplychain}. Given the increasing focus placed on this, as well as ideals of improving resource efficiency and developing the circular economy, it is essential to understand the material demands of products and processes. While LCA seeks to model the technosphere (the sum of all supply-chains), its focus is often on the environmental impacts of the system---the endpoints--- rather than the primary material flows themselves. 

A relatively new method, termed the crustal scarcity indicator (CSI)~\citep{arvidsson2020csi}, was developed to assess longterm global scarcity of minerals in life cycle assessment (LCA). This method introduced crustal scarcity potentials (CSPs) measured in kg silicon equivalents per kg element, derived from crustal concentrations. CSPs, provided for 76 elements, reflect the long-term global elemental scarcity based on crustal concentration proxies. The CSI, calculated by multiplying CSPs with extracted masses, effectively gauges the impact of elemental extraction. 

While useful for its stated purpose, the CSI presents its midpoint results in an abstract unit (kg-Si eq.) that is difficult to interpret and compare with other impact categories. Furthermore, the CSPs are not available for all elements, and the method does not allow for the quantification of material demands in terms of mass or volume.


\subsection{Waste in LCA}

Though often described simply as `a material with a negative economic value' \citep{guinee2004economicallocation}, waste is a nebulous concept, and one whose definition is poorly delineated and variable across space and time.  Moreover, from a systems perspective, the notion of waste is anathema to the circular economy, and it is far more useful to consider the identity and nature of the specific material flows. Thus, precise and detailed categorisation of these is essential to understand the `circularity' of an activity and its life cycle externalities. There is a conspicuous gap in the understanding of the waste footprint of human activities and their relationship with environmental damage \citep{laurenti2023wastefootprint}. Conventional LCAs consider waste as a `service' \citep{guinee2021wasteisnotaservice} and typically use generic waste processing models \citep{beylot2018} that break the causal link between the functional unit and the waste-associated impacts.

A significant portion of a product's total waste is generated during earlier stages such as resource extraction, transportation, and manufacturing, often remaining 'invisible' in traditional life cycle assessment (LCA) practices~\citep{laurenti2016wastefootprint}. This oversight in measuring and communicating a cradle-to-grave product waste footprint (PWF) highlights a gap in circular economy indicators. Traditional LCA does not typically view waste as having environmental significance by itself, focusing instead on emissions and resource use resulting from waste treatment. The environmental significance of waste and its correlation with other indicators has been a subject of extensive research. For example, studies have shown that popular resource footprints can cover a significant portion of environmental impact variance in product rankings~\citep{steinmann2017resourcefootprints}. However, correlations between various environmental indicators are not always consistent, as seen with the carbon footprint, which often does not correlate with other impact assessment scores~\citep{laurent2012carbonfootprint}. The aggregation of waste in PWFs raises concerns among LCA experts, regarding the uncertainties introduced by aggregated measures, as well as the potential misrepresentation of environmental performance due to differences in waste types~\citep{chen2021methoduncertainty,huijbregts2010energyfootprint}.

Moreover, existing LCA methodologies offer limited direct indicators at the impact assessment level, providing sparse information on the impacts of waste. This limitation becomes particularly evident when attempting to identify waste generation hotspots within a product's life cycle. Addressing these hotspots is crucial for advancing towards circularity. Thus, there is a pressing need for more comprehensive methods that can effectively quantify waste impacts and contribute to a better understanding of a product's total environmental footprint.
There is currently a lack of a convenient and flexible method for calculating waste flows in Life Cycle Assessment (LCA). 

\cite{laurenti2023wastefootprint} developed a method to calculate the waste footprint of a product or service based on solving the demand vectors of the activities, also presenting simple measures to quantify waste hazardousness and circularity. In that study, it was shown that the waste footprint correlates well with other LCIA methods, particularly human health. The method presented, however, is limited in its scope and flexibility, is computationally intensive, difficult to use, is not easily reproducible, and suffers from errors due to double counting. The WMF tool presented in the current work, builds on the work of Laurenti et.al., providing a more flexible, transparent and user-friendly approach to quantifying waste flows in LCA. Moreover, the WMF tool is not limited to waste, but can be used to quantify any supply-chain flow, such as water, gas, and critical raw materials. 


\subsection{The WasteAndMaterialFootprint Tool}

To better assess waste and material flows in LCA, the authors have developed a Python program built on the Brightway2 framework and designed to track these exchanges by translating them into indicators and impact categories. In this study, we present the WasteAndMaterialFootprint tool that enables LCA practitioners to easily aggregate the mass and volume of any desired exchange, and to creation of flexible categories to differentiate between waste types and End-of-Life (EOL) handling using (in this case) the Ecoinvent 3.9 cutoff database.

While methods with similar aims exist, they either lack flexibility and specificity \citep{foen2021ecofactors} or are cumbersome to apply and suffer from errors due to multiple counting \citep{laurenti2023wastefootprint}.

This tool provides a method for the calculation of waste footprint impact category results, differentiated by the type of waste handling. Furthermore, the tool facilitates rapid investigation and identification of waste hotspots, enabled by standard contribution analysis and Sankey diagram visualization tools. The authors consider this a crucial step in addressing the deficit of Life Cycle Assessment (LCA) methods that consider waste flows in the evaluation of a product or process' circular economy potential.





