% "As discussed in the introduction 1 there are some existing material demand methods in the standard LCIA method
% 264 sets, including the ‘crustal scarcity indicator’ (which provides only an aggregated, abstracted endpoint) (Arvidsson
% 265 et al., 2020) and the (deprecated) EDIP 2003 material use indicators (which provide endpoints in fundamental
% 266 units) (Hauschild and Potting, 2004). In these methods, the material demand is calculated based on the total mass that
% 267 is extracted from the environment, thus, their focus is essentially solely on the mining-related exchanges that bring
% 268 these materials from the biosphere into the technosphere."
\subsection{Background}\label{sec:intro-background}

The development of a `circular economy' has become a critical area of focus in the imperative pursuit of sustainability objectives and curtailment of our environmental footprint within planetary boundaries~\citep{eu2019greendeal, eu2020circ,nl2023ceplan,nl2016ceplan,pardo2018ce,ellenmacarthur2015ce}. Fundamental to this development is a decrease in primary material consumption and a reduction of life cycle waste through the implementation of `re-X' strategies (e.g., refuse, rethink, design for---and implementation of---repair, remanufacturing and recycling)~\citep{reike2018rex, eu2022ecodesign, eu2022repair,eu2015reman}. In addition to circular economy goals, contemporary geo-political tensions in an ever more globalised economy have highlighted the vulnerability of many advanced economies to intentional supply disruptions---wrought as an act of competition or even outright hostility~\citep{jrc2023supplychain,hartley2024cepolitics,berry2023crm}.

The concept of the `footprint' as an environmental sustainability indicator began with the Ecological Footprint (EF)~\citep{wackernagel1994ecologicalfootprint} and after being popularised by the Carbon Footprint (CF)~\citep{cucek2015environmentalfootprints} the `footprint family' developed strongly and has adopted many additional metrics---albeit without yet coalescing into a coherent or consistent framework~\citep{giampietro2014footprintstonowhere, vanham2019footprints,ridoutt2013footprints}. More recently, the footprint collection has been extended to include the Material Footprint (MF)~\citep{weidmann2013materialfootprint}, which is more often encountered in the `macro methods' of Industrial Ecology (IE), such as environmentally Extended Input-Output Analysis (EEIOA)~\citep{lenzen2022materialfootprint} and Material Flow Analysis (MFA)~\citep{schaffartzik2013mfafootprint}. Whether at the level of products, entire industries, nations or even continents, the material footprint aims to quantify the total material consumed in supply chains. It has been shown that MF can be `highly representative of damage to human health and biodiversity'~\citep{steinmann2017resourcefootprints} and indeed, this metric was recently beatified by the United Nations (UN), becoming the `core official indicator' for targets 8.4 and 12.2 of the Sustainable Development Goals (SDGs)~\citep{lenzen2022materialfootprint}.

The incipient Waste Footprint (WF) metric, in contrast, though emerging steadily, has yet to enjoy such noble recognition for its attempts to measure and classify the waste generated by human activities~\citep{laurenti2016wastefootprint}. Despite the fact that reducing life cycle waste is clearly critical to the development of the circular economy~\cite{towa2020wastefootprint,ellenmacarthur2015ce}, the WF remains largely overlooked---especially in LCA models where waste itself is seldom apportioned any inherent evironmental significance aside from the emissions related to its treatment~\citep{laurenti2023wastefootprint}.
Such ignominious neglect strikes the authors as unjustified, given that it has been repeatedly demonstrated that waste footprints can have a strong association with environmental damage~\citep{laurenti2023wastefootprint,doka2024publications, ridoutt2010wasteimpacts,jaio2013wasteabsorbtionfootprint}. Alas, in frequent, tragic failures of evironmental justice, it is often the most vunerable communities who suffer disproportionately from the social and ecological impacts that can be created by wasteful behaviours and sub-standard end-of-life treatment practices~\citep{pellow2023envjusticewaste,pellow2023envjusticewaste}.

WF and MF, the two metrics of focus in the study, can provide a comprehensive assessment of potential environmental impacts across the supply chain, encapsulating both resource use and pollution/waste generation, while offering insights at diverse scales, from individual activities to global systems, facilitating communication with a diverse range of stakeholders. Thus, to reduce the negative externalities of human consumption and improve supply chain resilience, it is essential to uncover, disaggregate, and quantify the material and waste footprints of human activities in as much detail as possible~\citep{bisinella2024wastelca, towa2020wastefootprint}.




Life Cycle Assessment (LCA) is a useful method for the holistic estimation of the environmental impacts of products and processes. LCA can comprehensively evaluate these impacts across the entire life cycle---from `cradle to grave'--- often identifying critical hotspots and guiding prioritisation of actions~\citep{guinee2011lca}. The standard approach is to apply Life Cycle Impact Assessment (LCIA) methods (such as ReCiPe~\citep{huijbregts2016recipe} and CML~\citep{guinee2002cml}), which convert the inventory data into a set of impact scores based on the sum of the elementary flows (those between the technosphere and biosphere). These scores can then be aggregated into a single score for each impact category, that can be compared across products and processes. Extending on standard LCA is \textit{ex-ante} LCA, which employs future scenarios to construct prospective background databases in an effort to predict the impacts of supply chains that have not yet been (and may never be) realised~\citep{cucurachi2018exante,blanco2020exante}.

Several LCIA methods include, to some extent, waste generation (Swiss Eco-Factors, EDIP and EN15804)~\citep{foen2021ecofactors,hauschild2003edip,cen2019en15804} and material consumption (Crustal Scarcity Indicator (CSI) and Swiss Eco-Factors~\citep{arvidsson2020csi,foen2021ecofactors}). These methods, however, are generally limited in their scope (especially for waste), do not allow for flexible quantification of specific waste and material types, and often provide results in characterised units that are abstract or difficult to interpret (e.g., Ümweltbelastungspunkte (UBP) in the case of the Swiss Eco-Factors)~\citep{su2020sustainableproddev}.


\subsubsection{Waste in LCA}\label{sec:intro-waste}

Though often described simply as a `material with a negative economic value'~\citep{guinee2004economicallocation}, waste is a nebulous concept, and one whose definition is poorly delineated and highly variable across space and time.  Moreover, from a systems perspective, the notion of waste is anathema to the `circular economy', they cannot co-exist. Returning from the abstract and etherial to the economic viability of the waste processing sector. While critical here are energy costs and the relative prices of virgin and secondary materials, the viability of an extensive waste processing system is highly dependent on precise knowledge (or at least, reasonable predictions) of material and waste flows. Thus, as~\cite{bisinella2024wastelca} argues, waste must be `thoroughly characterised' and that `modelling [of its management] must be physically based'. The reliable, robust models needed to guide the development of the circular economy can be built only on a foundation of high-quality data.

Accurate and detailed information about waste and waste systems is, by definition, essential for understanding the `circularity' of a human activity and predicting its life cycle externalities, but there remains a conspicuous knowledge gap regarding the waste footprints of human activities and their environmental impacts~\citep{laurenti2023wastefootprint}.


Conventional LCA database models consider waste as a `service' (accounting for the treatment, not the material)~\citep{guinee2021wasteisnotaservice} and typically use generic waste processing models~\citep{beylot2018} that break the causal link between the functional unit and the waste-associated impacts. In LCA, waste flows are (almost exclusively) not considered as fundamental biosphere exchanges, but rather, as technosphere flows within the human economy. Waste produced by an activity is transferred to a relevant waste treatment activity where it is accepted `burden-free' and transformed into a combination of emissions and other waste `products'~\citep{guinee2021wasteisnotaservice}. There can be several treatment steps in this pathway leading, ultimately, to a mass of material being deposited in a landfill. In this system of waste accounting, the impacts apportioned to the waste-producing activity are a sum of those incurred by the transport, treatment, and final disposal of the waste into terrestrial or aquatic environments. In particular, the extensive work of~\cite{doka2024publications} has contributed significantly to understanding the environmental impacts of waste treatment processes and the modelling the long-term impacts of disposal.

A considerable portion of a product's total waste is generated during earlier stages of its supply chain such as resource extraction, transportation, and manufacturing, thus, often remaining `invisible' to conventional LCA accounting practices~\citep{laurenti2016wastefootprint}. This oversight in measuring and communicating a cradle-to-grave product waste footprint (PWF) highlights a gap in circular economy indicators. Traditional LCA does not typically view waste as having environmental significance \textit{per se}, considering instead only emissions and resource use resulting from waste treatment~\cite{bisinella2024wastelca, laurenti2023wastefootprint}. The environmental significance of waste and its correlation with other indicators has been the subject of extensive research and studies have shown that popular resource footprints can cover a significant portion of environmental impact variance between activities~\citep{steinmann2017resourcefootprints,laurenti2023wastefootprint}. Correlations between various environmental indicators are not always consistent, however, as seen with the carbon footprint, which often does not correlate well with other impact assessment scores~\citep{laurenti2012carbonfootprint}. The aggregation of waste in PWFs raises concerns among LCA experts, regarding the uncertainties introduced by aggregated measures, as well as the potential misrepresentation of environmental performance due to differences in waste types~\citep{chen2021methoduncertainty,huijbregts2010energyfootprint}.

Existing LCA methodologies offer limited direct indicators at the impact assessment level, providing sparse information on the impacts of waste. This limitation becomes particularly evident when attempting to identify waste generation hotspots within a product's life cycle. Addressing these hotspots is crucial for advancing `circularity', but there is a lack of convenient and flexible ways to calculate waste flows in LCA and a pressing need for more comprehensive methods that can effectively quantify waste flows and, therefore, contribute to a greater understanding of a product's total environmental footprint.

\cite{laurenti2023wastefootprint} developed a method to calculate the waste footprint of a product or service based on solving the demand vectors of the activities, also presenting simple measures to quantify waste hazardousness and `circularity'. In that study, it was shown that the waste footprint correlates well with other LCIA methods, particularly human health. The method presented, however, is limited in its scope and flexibility, is computationally intensive, difficult to use, is not easily reproducible, and suffers from errors due to double counting (as the authors themselves acknowledge). The `T-reX' Python software tool presented herein provides a more flexible, transparent, and user-friendly approach to quantifying waste flows in LCA\@. Moreover, the utility of T-reX is not limited to waste, it can be used to categorise and aggregate any technosphere exchange of interest (or customised grouping thereof), such as water, gas, and critical raw materials.

\subsubsection{Material demand in LCA}\label{sec:intro-material}

In the context of a mineral-hungry renewable energy transition and constant geo-political tensions, more attention is being paid to the security of supply of materials, especially those considered `critical raw materials' (CRMs)~\citep{eu2023crmstudy,hool2023crm,mancini2013supplysecurity,jrc2023supplychain,hartley2024cepolitics,salviulo2021supplychain}. While LCA seeks to model the technosphere (a.k.a.\ the anthroposphere) and its exchanges with the biosphere, its focus is often on the environmental impacts of the system---the endpoints---rather than the primary material flows themselves.

A relatively new (2020) LCIA method, the Crustal Scarcity (CSI), was developed in order to introduce an assessment of the long-term global scarcity of minerals in LCA~\citep{arvidsson2020csi}. The CSI introduced crustal scarcity potentials (CSPs), which are measured in kg silicon equivalents per kg element and derived from crustal concentrations. CSPs, provided for 76 elements, reflect the long-term global elemental scarcity based on crustal concentration proxies. The CSI, calculated by multiplying CSPs with extracted masses, effectively gauges the impact of elemental extraction.

While useful for its intended purpose, the CSI presents its results in an abstract unit (kg-Si eq.) that is difficult to interpret and compare with other impact categories. Furthermore, the CSPs are not available for all elements (or more complex materials), and the method does not allow for the quantification of material demands in terms of mass or volume.

\subsection{Introduction to T-reX}\label{sec:intro-trex}

To facilitate the quantification of waste and material flows in LCA, we developed a Python package to extend the \texttt{brightway} LCA framework~\citep{mutel2017brightway} and designed to track these exchanges by translating them into inventory indicators and `pseudo' LCA impact (LCIA) categories. T-reX, in full, stands for `Tool for re-X' (reduce, reuse, recycle, etc.) and enables LCA practitioners to manipulate their databases such as to allow them to easily aggregate the mass and volume of any desired exchange, and to create flexible categories that differentiate between material categories, waste types, and EOL handling. While some methods with similar aims exist, they lack customisability and specificity~\citep{foen2021ecofactors} or can be cumbersome to apply and suffer from errors due to multiple counting~\citep{laurenti2023wastefootprint}.

Integration with the \texttt{premise} Python package~\citep{sacchi2022premise}---which connects the projections of integrated assessment models (IAMs) with current LCA databases---enables the user to easily create and manipulate prospective LCA databases. Frustratingly, the current utility of prospective databases---in general, and in particular for the waste sector---is constrained by the fact that, to date, the sectoral coverage of future life cycle inventories (LCIs) is largely confined to energy, steel, cement, and transport~\citep{sacchi2023premisedocs}. Indeed, despite the ever more critical need to reliably model and future waste management systems,~\cite{bisinella2024wastelca} reports an alarming lack of coherent development in this field.

The purpose of T-reX is not to quantify the environmental impacts of material consumption and waste production (the requsite data and impact modelling remains lacking), but rather to quantify the material and waste flows themselves, even those that are finally consumed by waste treatment processes. It provides, thus, not an impact assessment in the traditional sense, but an accounting of the material consumed and waste generated by a product or service inside of the technosphere, regardless of the end-of-life fate of these flows. By definition, the development of the `circular economy' necessitates the reduction and ultimate elimination of waste (though whether this objective is thermodynamically impossible has long been the subject of debate by~\cite{ayres1998recycling},~\cite{reuter2012recyclinglimits} and many others). In any case, minimising primary material consumption and the generation of waste is of paramount importance. By allowing LCA practitioners to easily classify and quantify these exchanges, T-reX provides a practical means to identify hotspots and highlight opportunities for waste reduction and material efficiency.




