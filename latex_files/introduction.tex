% "As discussed in the introduction 1 there are some existing material demand methods in the standard LCIA method
% 264 sets, including the ‘crustal scarcity indicator’ (which provides only an aggregated, abstracted endpoint) (Arvidsson
% 265 et al., 2020) and the (deprecated) EDIP 2003 material use indicators (which provide endpoints in fundamental
% 266 units) (Hauschild and Potting, 2004). In these methods, the material demand is calculated based on the total mass that
% 267 is extracted from the environment, thus, their focus is essentially solely on the mining-related exchanges that bring
% 268 these materials from the biosphere into the technosphere."
\subsection{Background}

The development of a `circular economy' has become a critical area of focus in the imperative pursuit of sustainability objectives and curtailment of our environmental footprint within planetary boundaries~\citep{eu2019greendeal, eu2020circ,nl2023ceplan,nl2016ceplan,pardo2018ce,ellenmacarthur2015ce}. Fundamental to this development is a decrease in primary material consumption and a reduction of life cycle waste through the implementation of `re-X' strategies (e.g., refuse, rethink, design for---and implementation of---repair, remanufacturing and recycling)~\citep{eu2022ecodesign, eu2022repair,eu2015reman}. In addition to circular economy goals, contemporary geo-political tensions in an ever more globalised economy have highlighted the vulnerability of many advanced economies to intentional supply disruptions, wrought as an act of competition or outright hostility~\citep{jrc2023supplychain,hartley2024cepolitics,berry2023crm}.

While some material demands are apparent in the final product and the waste generated may be inferred from knowledge of the use and end-of-life (EOL) phases, a significant proportion of these can be `hidden' in the supply chain and thus not reported directly in the final results~\citep{laurenti2016wastefootprint,salviulo2021supplychain}. The concept of the 'footprint' as an environmental sustainability indicator began with the Ecological Footprint (EF)~\citep{wackernagel1994ecologicalfootprint} being since popularised by the Carbon Footprint (CF) and more recently extended to include the Material Footprint (MF) and Waste Footprint (WF)~\cite{cucek2015environmentalfootprints}. It has been found that these material footprints can be `highly representative of damage to human health and biodiversity'~\citep{steinmann2017resourcefootprints} and that waste footprints have a `strong association' with environmental damage~\citep{laurenti2023wastefootprint}. Thus, to reduce the negative externalities of consumption and improve supply chain resilience, it is essential to uncover, disaggregate, and quantify the material and waste footprints of human activities in as much detail as possible.

Life Cycle Assessment (LCA) is a useful method for the holistic estimation of the environmental impacts of products and processes. LCA can comprehensively evaluate these impacts across the entire life cycle---from `cradle to grave'---, often identifying critical hotspots and guiding prioritisation of actions. The standard approach is to apply Life Cycle Impact Assessment (LCIA) methods (such as ReCiPe~\citep{huijbregts2016recipe} and CML~\citep{guinee2002cml}), which convert the inventory data into a set of impact scores based on the sum of the elementary flows. These scores are then aggregated into a single score for each impact category, which can be compared across products and processes.

Several LCIA methods include, to some extent, waste generation (Swiss Eco-Factors, EDIP and EN15804)~\citep{foen2021ecofactors,hauschild2003edip,cen2019en15804} and material consumption (Crustal Scarcity Indicator (CSI) and Swiss Eco-Factors~\citep{arvidsson2020csi,foen2021ecofactors}). These methods, however, are generally limited in their scope (especially for waste), do not allow for flexible quantification of specific waste and material types, and often provide results in characterised units that are abstract or difficult to interpret (e.g., Ümweltbelastungspunkte (UBP) in the case of the Swiss Eco-Factors). [ADD CITATION HERE]


\subsubsection{Waste in LCA}

Though often described simply as a `material with a negative economic value'~\citep{guinee2004economicallocation}, waste is a nebulous concept, and one whose definition is poorly delineated and highly variable across space and time.  Moreover, from a systems perspective, the notion of waste is anathema to the `circular economy', they cannot co-exist. Returning from the abstract, the economic viability of the waste processing sector depends on precise knowledge (or reasonable predictions of material/waste flows), thus, as ~\cite{bisinella2024wastelca} argues, waste must be `thoroughly characterised' and that `modelling [of its management] must be physically based'. 

Accurate and detailed information about waste and waste systems is, by definition, essential for understanding the `circularity' of an activity and predicting its life cycle externalities, but there is a conspicuous knowledge gap regarding the waste footprints of human activities and their environmental impacts~\citep{laurenti2023wastefootprint}. 


Conventional LCAs consider waste as a `service'~\citep{guinee2021wasteisnotaservice} and typically use generic waste processing models~\citep{beylot2018} that break the causal link between the functional unit and the waste-associated impacts. wIn LCA, waste flows are not considered as fundamental biosphere exchanges, but rather as technosphere flows. Waste produced by an activity is transferred to a relevant waste treatment activity where it is accepted `burden-free' and transformed into a combination of emissions and other waste `products'~\citep{guinee2021wasteisnotaservice}. There can be several treatment steps in this pathway leading, ultimately, to a mass of material being deposited in a landfill. In this system of waste accounting, the impacts apportioned to the waste-producing activity are a sum of those incurred by the transport, treatment, and final disposal of the waste into terrestrial or aquatic environments. In particular, the extensive work of~\cite{doka2024publications} has contributed significantly to understanding the environmental impacts of waste treatment processes and the long-term impacts of disposal.

A significant portion of a product's total waste is generated during earlier stages such as resource extraction, transportation, and manufacturing, often remaining `invisible' in traditional LCA practices~\citep{laurenti2016wastefootprint}. This oversight in measuring and communicating a cradle-to-grave product waste footprint (PWF) highlights a gap in circular economy indicators. Traditional LCA does not typically view waste as having environmental significance by itself, focusing instead on emissions and resource use resulting from waste treatment~\cite{bisinella2024wastelca, laurenti2023wastefootprint}. The environmental significance of waste and its correlation with other indicators has been the subject of extensive research and studies have shown that popular resource footprints can cover a significant portion of environmental impact variance between activities~\citep{steinmann2017resourcefootprints}. However, correlations between various environmental indicators are not always consistent, as seen with the carbon footprint, which often does not correlate well with other impact assessment scores~\citep{laurenti2012carbonfootprint}. The aggregation of waste in PWFs raises concerns among LCA experts, regarding the uncertainties introduced by aggregated measures, as well as the potential misrepresentation of environmental performance due to differences in waste types~\citep{chen2021methoduncertainty,huijbregts2010energyfootprint}.

Moreover, existing LCA methodologies offer limited direct indicators at the impact assessment level, providing sparse information on the impacts of waste. This limitation becomes particularly evident when attempting to identify waste generation hotspots within a product's life cycle. Addressing these hotspots is crucial for advancing towards circularity, however,  there is a lack of a convenient and flexible way to calculate waste flows in LCA and a pressing need for more comprehensive methods that can effectively quantify waste flows and, therefore, contribute to a better understanding of a product's total environmental footprint. 

\cbox{Laurenti's latest. I'm not sure how to move it earlier...}

\cite{laurenti2023wastefootprint} developed a method to calculate the waste footprint of a product or service based on solving the demand vectors of the activities, also presenting simple measures to quantify waste hazardousness and circularity. In that study, it was shown that the waste footprint correlates well with other LCIA methods, particularly human health. The method presented, however, is limited in its scope and flexibility, is computationally intensive, difficult to use, is not easily reproducible, and suffers from errors due to double counting. The T-reX package presented herein provides a more flexible, transparent, and user-friendly approach to quantifying waste flows in LCA. Moreover, the T-reX tool is not limited to waste but can be used to categorise and aggregate any technosphere exchange of interest (or customised grouping thereof), such as water, gas, and critical raw materials. 

\subsubsection{Material demand in LCA}

In the context of a mineral-hungry renewable energy transition and recent geo-political tensions, more attention is being paid to the security of supply of materials, especially those considered `critical raw materials' (CRMs)~\citep{eu2023crmstudy,hool2023crm,mancini2013supplysecurity,jrc2023supplychain,hartley2024cepolitics,salviulo2021supplychain}. While LCA seeks to model the technosphere (a.k.a.\ the anthroposphere), its focus is often on the environmental impacts of the system---the endpoints---rather than the primary material flows themselves. 

A relatively new (2020) LCA method, the CSI, was developed to assess long-term global scarcity of minerals. The CSI introduced crustal scarcity potentials (CSPs), which are measured in kg silicon equivalents per kg element and derived from crustal concentrations. CSPs, provided for 76 elements, reflect the long-term global elemental scarcity based on crustal concentration proxies. The CSI, calculated by multiplying CSPs with extracted masses, effectively gauges the impact of elemental extraction. 

While useful for its stated purpose, the CSI presents its midpoint results in an abstract unit (kg-Si eq.) that is difficult to interpret and compare with other impact categories. Furthermore, the CSPs are not available for all elements (or more complex materials), and the method does not allow for the quantification of material demands in terms of mass or volume.

\subsubsection{Introduction to the T-reX package}

To facilitate quantification of waste and material flows in LCA, we have developed a Python program built on the Brightway framework~\citep{mutel2017brightway} and designed to track these exchanges by translating them into indicators and `pseudo' LCA impact (LCIA) categories. T-reX enables LCA practitioners to easily manipulate their databases to allow them to easily aggregate the mass and volume of any desired exchange, and to create flexible categories that differentiate between material categories, waste types, and EOL handling. While methods with similar aims exist, they lack customisability and specificity~\citep{foen2021ecofactors} or can be cumbersome to apply and suffer from errors due to multiple counting~\citep{laurenti2023wastefootprint}.

\cbox{Prospective databases}

Integration with the \texttt{premise} package~\citep{sacchi2022premise}---which connects the projections of integrated assessment models (IAMs) with current LCA databases---enables the user to easily create and manipulate prospective LCA databases. The current utility of prospective databases (in general, and in particular for the waste sector) is constrained by the fact that, to date, the sectoral coverage of future life cycle inventories (LCIs) is largely confined to energy, steel, cement, and transport~\citep{sacchi2023premisedocs}. Indeed, despite the ever more critical need model and future waste management systems, \cite{bisinella2024wastelca} reports an alarming lack of 


\cbox{Purpose of the T-reX, conclusion}


The purpose of the T-reX tool is not to quantify the environmental impacts of material consumption and waste production, but rather to quantify the material and waste flows themselves, even those that are finally consumed by waste treatment processes. It provides, thus, not an impact assessment in the traditional sense, but an accounting of the material consumed and waste generated by a product or service inside of the technosphere, regardless of the end-of-life fate of these flows. By definition, the development of the `circular economy' necessitates the reduction and ultimate elimination of waste---though whether this objective is thermodynamically impossible has long been the subject of debate by~\cite{ayres1998recycling},~\cite{reuter2012recyclinglimits} and many others. In any case, avoiding material consumption and generation of waste is of paramount importance. By allowing LCA practitioners to easily classify and quantify these exchanges, the T-reX tool provides a practical means to identify hotspots and opportunities for waste reduction and material efficiency.




