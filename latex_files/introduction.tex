\subsection{Introduction to T-reX}\label{sec:intro-trex}

Development of a `circular economy' has become a central focus in the frantic pursuit of sustainability objectives toward curtailment of our environmental footprint within planetary boundaries~\citep{eu2019greendeal, eu2020circ,nl2023ceplan,nl2016ceplan,pardo2018ce,ellenmacarthur2015ce}. Fundamental to this development is a decrease in primary material consumption and a reduction of life cycle waste through the implementation of `re-X' strategies (e.g., refuse, rethink, design for---and implementation of---repair, remanufacturing and recycling)~\citep{reike2018rex, eu2022ecodesign, eu2022repair, eu2015reman}. In addition to circular economy goals, contemporary geopolitical tensions in an ever more globalised economy have highlighted the vulnerability of many advanced economies to intentional supply disruptions---wrought as an act of competition or even outright hostility~\citep{jrc2023supplychain,hartley2024cepolitics,berry2023crm}.

The concept of the `footprint' as an environmental sustainability indicator began with the Ecological Footprint (EF)~\citep{wackernagel1994ecologicalfootprint} and after being popularised by the Carbon Footprint (CF)~\citep{cucek2015environmentalfootprints} the `footprint family' has adopted many additional metrics---albeit without yet coalescing into a coherent or consistent framework~\citep{giampietro2014footprintstonowhere, vanham2019footprints,ridoutt2013footprints}. More recently, the footprint collection has been extended to include the Material Footprint (MF)~\citep{weidmann2013materialfootprint}, which is often encountered in the `macro methods' of Industrial Ecology (IE), such as environmentally Extended Input-Output Analysis (EEIOA)~\citep{lenzen2022materialfootprint} and Material Flow Analysis (MFA)~\citep{schaffartzik2013mfafootprint}. Whether at the level of products, entire industries, nations, or even continents, the material footprint aims to quantify the total material consumed in supply chains. It has been shown that the MF can be `highly representative of damage to human health and biodiversity'~\citep{steinmann2017resourcefootprints} and indeed, this metric was recently beatified by the United Nations (UN), becoming the `core official indicator' for targets 8.4 and 12.2 of the Sustainable Development Goals (SDGs)~\citep{lenzen2022materialfootprint}.

Though steadily emerging~\citep{laurenti2016wastefootprint,demirer2019wastefootprint,guillotreau2023wastefootprint}, the `Waste Footprint' (WF) metric is less well-developed. While it seems obvious that reducing life cycle waste is critical to the development of the circular economy~\citep{towa2020wastefootprint,ellenmacarthur2015ce}, the WF remains largely overlooked---especially in LCA models where waste itself is seldom apportioned any inherent environmental significance aside from the emissions related to its treatment~\citep{laurenti2023wastefootprint}. Such neglect strikes the authors as unjustified, given that it has been repeatedly demonstrated that WFs have a strong association with environmental damage~\citep{laurenti2023wastefootprint,doka2024publications, ridoutt2010wasteimpacts,jaio2013wasteabsorbtionfootprint}. Additionally, it has been shown that the most vulnerable communities suffer disproportionately from the social and ecological impacts created by wasteful behaviours and substandard end-of-life treatment~\citep{pellow2023envjusticewaste,akese2018envjustice}.

WF and MF, the two metrics of focus in the study, can provide a comprehensive assessment of potential environmental impacts across the supply chain, encapsulating both resource use and pollution/waste generation, while offering insights at diverse scales, from individual activities to global systems, facilitating communication with a diverse range of stakeholders. Thus, to reduce the negative externalities of human consumption and improve supply chain resilience, it is essential to uncover, disaggregate, and quantify the material and waste footprints of human activities in as much detail as possible~\citep{bisinella2024wastelca, towa2020wastefootprint,berger2020mineralsinlca,sonderegger2020mineralsinlca}.


Life Cycle Assessment (LCA) is a useful method for the holistic estimation of the environmental impacts of products and processes. LCA can comprehensively evaluate impacts across the entire life cycle---from `cradle to grave'--- often identifying critical hotspots and guiding prioritisation of action~\citep{guinee2011lca}. The standard approach is to apply Life Cycle Impact Assessment (LCIA) methods (such as ReCiPe~\citep{huijbregts2016recipe} and CML~\citep{guinee2002cml}), which convert the inventory data into a set of impact scores based on the sum of the elementary flows (those between the technosphere and biosphere). These scores can then be aggregated into one metric for each impact category that can be compared across products and processes. Extending on standard LCA is \textit{ex-ante} LCA, which employs future scenarios to construct `prospective background databases' to predict the impacts of supply chains that have not yet been (and may never be) realised~\citep{cucurachi2018exante,blanco2020exante}.

Several LCIA methods include, to some extent, waste generation (Swiss Eco-Factors, EDIP and EN15804)~\citep{foen2021ecofactors,hauschild2003edip,cen2019en15804} and material consumption (Crustal Scarcity Indicator (CSI) and Swiss Eco-Factors~\citep{arvidsson2020csi,foen2021ecofactors}). These methods, however, are generally limited in their scope (especially for waste), do not allow for flexible quantification of specific waste and material types, and often provide results in characterised units that are abstract or difficult to interpret (e.g., \textit{Ümweltbelastungspunkte} (UBP) [English: environmental impact points] in the case of the Swiss Eco-Factors)~\citep{su2020sustainableproddev}.


To facilitate the quantification of waste and material flows in LCA, we developed a Python package to extend the \texttt{brightway} open-source LCA framework~\citep{mutel2017brightway} and track these exchanges by translating them into inventory indicators and `pseudo' LCA impact (LCIA) categories. T-reX is `T(ool for) re-X' (reduce, reuse, recycle, etc...) and enables LCA practitioners to manipulate their databases to allow them to easily aggregate the mass and volume of any desired exchange, and to create flexible categories that differentiate between material categories, waste types, and treatment or EOL handling.

Integration with the \texttt{premise} Python package~\citep{sacchi2022premise}---which connects the projections of integrated assessment models (IAMs) with current LCA databases---enables the user to create and manipulate prospective LCA databases. Frustratingly, the current utility of prospective databases---in general, and in particular for the waste sector---is constrained by the fact that, to date, the sectoral coverage of future life cycle inventories (LCIs) is largely confined to energy, steel, cement, and transport~\citep{sacchi2023premisedocs}. Despite the ever more critical need to reliably model and future waste management systems,~\cite{bisinella2024wastelca} reports an alarming lack of coherent development in this field. \texttt{premise} offers an excellent structure for the integration of future waste and material flow modelling in LCA databases, but its potential contribution to the development of the circular economy cannot be realised if the scientific resources for data collection and validation are not forthcoming. The authors believe that there is an urgent need for research investment in this area.

The purpose of T-reX is not to quantify the environmental impacts of material consumption and waste production (the requisite data and impact modelling remain lacking), but rather to quantify the material and waste flows themselves, even those that are finally consumed by waste treatment processes. It provides, thus, not an impact assessment in the traditional sense, but an accounting of the material consumed and waste generated by a product or service inside of the technosphere, regardless of the end-of-life fate of these flows. By definition, the development of the `circular economy' necessitates the reduction and ultimate elimination of waste (though whether this objective is thermodynamically impossible has long been the subject of debate by~\cite{ayres1998recycling},~\cite{reuter2012recyclinglimits} and many others). In any case, minimising primary material consumption and the generation of waste is of paramount importance. By allowing LCA practitioners to easily classify and quantify these exchanges, T-reX provides a practical means to identify hotspots and potentially highlight opportunities for waste reduction and material efficiency.

\subsection{Background and need}\label{sec:intro-background}
\subsubsection{Waste in LCA}\label{sec:intro-waste}

Though often described simply as a `material with a negative economic value'~\citep{guinee2004economicallocation}, waste is a nebulous concept, and one whose definition is poorly delineated and highly variable across space and time. From a systems perspective, the notion of waste is anathema to the `circular economy', they cannot co-exist. From a practical economic viewpoint, the viability of an extensive waste processing system is highly dependent on precise knowledge (or at least, reasonable predictions) of material and waste flows, in addition to energy costs and the relative prices of virgin and secondary materials. Thus, as~\cite{bisinella2024wastelca} argues, waste must be `thoroughly characterised' and `modelling [of its management] must be physically based'. The reliable, robust models needed to guide the development of the circular economy can be built only on a foundation of high-quality data.

Accurate and detailed information about waste and waste systems is, by definition, essential for understanding the `circularity' of a human activity and for predicting its life cycle externalities. There remains, however, a knowledge gap regarding WFs and their associated environmental impacts~\citep{laurenti2023wastefootprint}.

Conventional LCA database models consider waste as a `service' (accounting for the treatment, not the material)~\citep{guinee2021wasteisnotaservice} and typically use generic waste processing models~\citep{beylot2018} that break the causal link between the functional unit and the waste-associated impacts. In LCA, waste flows are (almost exclusively) not considered as fundamental biosphere exchanges, but rather, as technosphere flows within the human economy. Waste produced by an activity is transferred to a relevant waste treatment activity where it is accepted `burden-free' and transformed into a combination of emissions and other waste `products'~\citep{guinee2021wasteisnotaservice}. There can be several treatment steps in this pathway leading, ultimately, to a mass of material being deposited in a landfill. In this system of waste accounting, the impacts apportioned to the waste-producing activity are a sum of those incurred by the transport, treatment, and final disposal of the waste into terrestrial or aquatic environments. In particular, the extensive work of~\cite{doka2024publications} has contributed significantly to understanding the environmental impacts of waste treatment processes and the modelling of the long-term impacts of disposal.

A considerable portion of a product's total waste is generated during earlier stages of its supply chain such as resource extraction, transportation, and manufacturing, thus, often remaining `invisible' to conventional LCA accounting practices~\citep{laurenti2016wastefootprint}. This oversight in measuring and communicating a cradle-to-grave product waste footprint (PWF) highlights a gap in circular economy indicators. Traditional LCA does not typically view waste as having environmental significance \textit{per se}, considering instead only emissions and resource use resulting from waste treatment~\citep{bisinella2024wastelca, laurenti2023wastefootprint}. The environmental significance of waste and its correlation with other indicators has been the subject of extensive research. Studies have shown that popular resource footprints can cover a significant portion of environmental impact variance between activities~\citep{steinmann2017resourcefootprints,laurenti2023wastefootprint}. Correlations between various environmental indicators are not always consistent, however, as seen with the carbon footprint, which often does not correlate well with other impact assessment scores~\citep{laurenti2012carbonfootprint}. The aggregation of waste in PWFs raises concerns among LCA experts, regarding the uncertainties introduced, as well as the potential misrepresentation of environmental performance due to differences in waste types~\citep{chen2021methoduncertainty,huijbregts2010energyfootprint}.

Some existing LCA methodologies offer direct indicators at the impact assessment level, but generally, there is little attention given to the impacts of waste in LCA~\citep{lauran2020abioticdepletion}. This limitation becomes particularly evident when attempting to identify waste generation hotspots within a product's life cycle. Addressing these hotspots is crucial for advancing `circularity', but there is a lack of convenient and flexible methods to calculate waste flows in LCA and a pressing need for more comprehensive methods that can effectively quantify waste flows and, therefore, contribute to a greater understanding of a product's total environmental footprint.

~\cite{laurenti2023wastefootprint} developed a method to calculate the waste footprint of a product or service based on solving the demand vectors of the activities, while also proposing simple measures to quantify waste `hazardousness' and `circularity' with this data. The method presented, as the authors explain, suffers from errors due to double counting due to the way that exchanges are identified and `tagged' . In T-reX, the source database is deconstructed into a list of separate exchanges, and only those explicitly defined as hazardous are marked as such. The `T-reX' Python software tool presented herein provides a more flexible, transparent, and user-friendly approach to quantifying waste flows in LCA\@. Moreover, the utility of T-reX is not limited to waste, it can be used to categorise and aggregate any technosphere exchange of interest (or customised grouping thereof), such as water, gas, and critical raw materials.

\subsubsection{Material demand in LCA}\label{sec:intro-material}

In the context of geopolitical tension and a mineral-hungry `renewable energy transition', greater attention is being given to the security of material supply---especially for those considered `critical raw materials' (CRMs)~\citep{eu2023crmstudy,hool2023crm,mancini2013supplysecurity,jrc2023supplychain,hartley2024cepolitics,salviulo2021supplychain,iea2023crm,iea2023energytechperspectives}. While LCA seeks to model the technosphere (a.k.a.\ the anthroposphere) and its exchanges with the biosphere, its focus is often on the environmental impacts of the system rather than the primary material flows themselves.

The Crustal Scarcity (CSI) LCIA method was developed in order to introduce an assessment of the long-term global scarcity of minerals in LCA~\citep{arvidsson2020csi}. The CSI introduced crustal scarcity potentials (CSPs), which are measured in kg--silicon--equivalents per kg--element and derived from crustal concentrations. CSPs, provided for 76 elements, reflect the long-term global elemental scarcity based on crustal concentration proxies. The CSI, calculated by multiplying CSPs with extracted masses, effectively gauges the impact of elemental extraction.

While useful for its intended purpose, the CSI presents its results in an abstract unit (kg-Si eq.) that is difficult to interpret and compare with other impact categories. Furthermore, the CSPs are not available for all elements (or more complex materials), and the method does not allow for the quantification of material demands in terms of mass or volume.