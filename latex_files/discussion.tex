Given that both waste generation and material demand are often strongly associated with the environmental impacts of an activity, it is important that they are included in the LCA. While there are numerous examples of existing and proposed methods that attempt to provide endpoint LCIA scores through convoluted formulae or subjective weighting, there is little consensus on their application and their complexity and lack of transparency can make them difficult to use and interpret.

Our contribution advances the state-of-the-art by giving LCA practitioners a simple, flexible, and transparent way to calculate supply chain waste and material footprints, delivering results in standard units and as direct aggregations of the relevant demand inventories. Once the databases have been processed with the T-reX tool, the user can  easily apply the T-reX `pseudo-LCIA' methods to calculate the waste and material inventory footprints in the same way that they would with a conventional LCIA method.

The simple case study of five Li-ion batteries presented in this paper demonstrated the the utility, flexibility and limitations of the T-reX tool. 

First, by adjusting the user configuration for future scenarios and waste/material categorisation we easily produced a set of customised versions of the ecoinvent database. Then, by applying the T-reX `pseudo-LCIA' methods, it was trivial to calculate categorised waste and material inventory footprints for a number of present and future supply chains. Additionally, visual exploration in \texttt{ActivityBrowser} was possible because the T-reX `pseudo-LCIA' methods are integrated in the \texttt{brightway} project in the same way as standard LCIA methods.

One limitation of the T-reX tool is that it does not yet provide specific information (in a readily accessible format) on the composition of the waste generated. This is the information would be needed to thoroughly assess the potential environmental impacts of this waste. Currently, the user would need to manually explore the waste footprint inventory produced by the application of the T-reX to determine if, for example, the waste generated represents an actual loss of resources, or is simply a transfer of the `overburden' in mining activity, which is classified as `inert waste'. A methodic classification of waste exchanges and the end-of-life fates will be facilitated by the more detailed and disaggregated data that is seen in each successive release of ecoinvent~\citep{fitzgerald2023ecoinventdocumentation}.

The utility of the T-reX tool in studies of future supply chains is limited by the fact that the currently available prospective databases focus largely on changes in the energy, steel, cement, and transport sectors~\citep{sacchi2023premisedocs}. As demonstrated in the results of the case study---where there was often very little scenario-temporal change in many waste and material footprint indicators---the utility of prospective LCA is restricted if there is little adaption of the future background inventories. In particular, the inclusion of scenarios with future waste processing technology would greatly improve our predictions of waste and material flows. A strong focus on enhancing these prospective databases is, thus, of critical importance to the future of prospective LCA, and by extension, to the development of the circular economy. 
