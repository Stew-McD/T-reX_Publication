Given that waste generation and material demand are often strongly associated with the environmental impacts of human activities~\citep{laurenti2023wastefootprint,steinmann2017resourcefootprints, demirer2019wastefootprint}, we consider it of high importance that they are included LCA accounting. Although there are existing LCIA methods that provide endpoint impact scores related to material demand and waste generation, they generally contain convoluted formulae or subjective weighting, or their complexity and lack of transparency can make them difficult to use and interpret~\citep{foen2021ecofactors,hauschild2003edip,cen2019en15804, arvidsson2020csi,foen2021ecofactors}.

T-reX advances the state-of-the-art in LCA providing practitioners with a simple, flexible, and transparent way to calculate supply chain waste and material footprints, delivering results in standard units and as direct aggregations of the relevant demand inventories. Once LCA databases in the \texttt{brightway} project have been processed with T-reX, the user can easily apply the T-reX `pseudo-LCIA' methods to calculate the waste and material inventory footprints in the same way that they would with a conventional LCIA method.

The simple case study of five Li-ion batteries presented in this paper demonstrated the  utility, flexibility as well as some limitations of T-reX.

First, by adjusting the user configuration for future scenarios and waste/material categorisation we produced a set of customised versions of current and prospective ecoinvent databases along with a `pseudo-biosphere' T-reX database. Then, by applying the T-reX `pseudo-LCIA' methods, categorised waste and material inventory footprints for a number of present and future supply chains were trivial to calculate. Additionally, visual exploration in \texttt{ActivityBrowser} was possible because the T-reX `pseudo-LCIA' methods are integrated within the \texttt{brightway} project in the same way as standard LCIA methods.

One current limitation of T-reX is that it does not yet provide specific information (in a readily accessible format) on the composition of the waste generated. This is the information that would be needed to thoroughly assess the potential environmental impacts of this waste. Currently, the user would need to manually explore the waste footprint inventory produced by the application of the T-reX to determine if the waste generated represents an actual loss of resources or environmental risk, or, for example, is simply a transfer of the `overburden' in mining activity, which is classified as `inert waste'. A methodic classification of waste exchanges and the end-of-life fates is expected to be facilitated by the ever more detailed and disaggregated data that is seen in each successive release of ecoinvent~\citep{fitzgerald2023ecoinventdocumentation}.

The utility of T-reX in studies of future supply chains is limited by the fact that the currently available prospective databases focus largely on changes in the energy, steel, cement, and transport sectors~\citep{sacchi2023premisedocs}. As demonstrated in the results of the case study---where there was often very little scenario-temporal change in many waste and material footprint indicators---the utility of prospective LCA is restricted if there is inadequate adaption of the future background inventories. In particular, the inclusion of scenarios with future waste processing technology could greatly improve our predictions of waste and material flows and offer valuable insight into their potential impacts~\citep{bisinella2024wastelca}. A strong focus on enhancing these prospective databases is, thus, of critical importance to the future of prospective LCA, and by extension, to the development of the `circular economy'. 
