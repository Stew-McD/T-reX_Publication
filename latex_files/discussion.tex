Given that both waste generation and material demand are often strongly associated with the environmental impacts of an activity, it is important that they are included in the LCA. While there are numerous examples of existing and proposed methods that attempt to provide endpoint LCIA scores through convoluted formulae or subjective weighting, there is little consensus on their application and their complexity and lack of transparency can make them difficult to use and interpret.

The WMF tool, in contrast, gives LCA practitioners a simple, flexible, and transparent way to calculate the supply chain waste and material footprints, delivering results in standard units as direct aggregations of the relevant demand inventories. Once the databases have been processed with the WMF tool, the user can then easily apply the WMF methods to calculate the supply-chain waste and material footprints for any activity in the same way that they would calculate any LCIA indicator.

The simple case study presented in this paper demonstrates both the utility and the limitations of the WMF tool. It was shown the WMF tool was able to calculate categorised aggregates and contribution analyses of both the waste generated and material demands in the present and future supply chains of the Li-ion batteries under consideration. Further, integration of the WMF methods as `pseudo-LCIA' methods allows the user to easily make use of the WMF tool in their preferred LCA software, be it code-based like \texttt{brightway2} or graphical like \texttt{ActivityBrowser}.

One main limitation of the WMF tool is that it does not yet provide specific information (in a readily accessible format) on the composition of the waste generated, which would be needed to thoroughly assess the environmental impacts of this waste. Currently, the user would need to manually explore the `waste inventory' produced by the application of the WMF to determine if, for example, the waste generated represents an actual loss of resources, or is simply a transfer of the `overburden' in mining activity, which is classified as `inert waste'. A methodic classification of waste exchanges and the end-of-life fates will be facilitated by the more detailed and disaggregated data that is seen in each successive release of ecoinvent~\citep{fitzgerald2023ecoinventdocumentation}.

Furthermore, the utility of the WMF tool in studies of future supply chains is limited by the fact that the currently available prospective databases focus largely on changes in the energy, steel, cement, and transport sectors~\citep{sacchi2023premisedocs}. As demonstrated in the results of the case study---where there was often very little scenario-temporal change in many waste and material footprint indicators---the utility of prospective LCA is restricted if there is little adaption of the future background inventories. In particular, the inclusion of scenarios with future waste processing technology would greatly improve our predictions of waste and material flows. A strong focus on enhancing these prospective databases is, thus, of critical importance to the future of prospective LCA, and by extension, to the development of the circular economy. 
